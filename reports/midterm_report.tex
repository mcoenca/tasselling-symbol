\documentclass[12pt]{article}
	\usepackage{amsmath}
	\usepackage{amssymb}
	\usepackage{fancyhdr}
	\usepackage{float}
	\usepackage{graphicx}
	\usepackage{cite}

	\oddsidemargin0cm
	\topmargin-2cm     %I recommend adding these three lines to increase the 
	\textwidth16.5cm   %amount of usable space on the page (and save trees)
	\textheight23.5cm  

\newcommand{\myname}{Evan Palmer, Titouan Rigoudy}
\newcommand{\myandrew}{esp@andrew, trigoudy@andrew}
\newcommand{\myhwnum}{1}
\newcommand{\problemnum}{1}
\newcommand{\thedate}{\today}
\DeclareMathOperator*{\argmax}{arg\,max}
%Page header
	\setlength{\parindent}{0pt}
	\setlength{\parskip}{5pt plus 1pt}
	 
	\pagestyle{fancyplain}
	\lhead{\fancyplain{}{\textbf{Midway report}}}      % Note the different brackets!
	\rhead{\fancyplain{}{\myname\\ \myandrew}}
	\chead{\fancyplain{}{10-701}}
\begin{document}
%Title
	\medskip    
	\thispagestyle{plain}
	\begin{center}                 
	{\LARGE Finding The Best Critic For You} \\
	\medskip
	Machine Learning Midway Report \\
	\smallskip
	\myname \\
	\myandrew \\
	\thedate \\
	\end{center}
	\vspace{0.5cm}

\section{Introduction}


\section{Obtaining the data}



\section{Data descriptions}

\subsection{Rotten tomatoes}

	The data we retrieved from Rotten tomatoes included approximately five hundred thousand reviews by about four thousand unique critics about four thousand movies.


	\begin{table}[H]
	 \centering
	 \caption{Distribution of number of reviews per critic for movies on rotten tomatoes}
	 \begin{tabular}{ l | c | c | c | c }
	 \hline
	 &  Min & Max & Mean & Std Dev  \\
	 \hline
	 Top Critcs & 0 & 56 & 22.41 & 16.06 \\
	 Other Critics & 0 & 316 & 92.24 & 68.12 \\
	 \hline
	 \end{tabular}
	 \end{table}

	\begin{figure}[H]
	    \centering
	    \includegraphics[width=0.48\textwidth]{plots/plot_r_mov_top.png}
	    \includegraphics[width=0.48\textwidth]{plots/plot_r_mov_oth.png}
	    \caption{INSERT TITLE}
	    \label{fig:r_mov}
	\end{figure}


	\begin{table}[H]
	 \centering
	 \caption{Distribution of number of reviewed movies per critic on rotten tomatoes} 
	 \begin{tabular}{ l | c | c | c | c }
	 \hline
	 &  Min & Max & Mean & Std Dev  \\
	 \hline
	 Top Critcs & 0 & 2862 & 21.79 & 124.96 \\
	 Other Critics & 0 & 2634 & 68.16 & 224.77 \\
	 \hline
	 \end{tabular}
	 \end{table}

	\begin{figure}[H]
	    \centering
	    \includegraphics[width=0.48\textwidth]{plots/plot_r_crit_top.png}
	    \includegraphics[width=0.48\textwidth]{plots/plot_r_crit_oth.png}
	    \caption{INSERT TITLE}
	    \label{fig:r_crit}
	\end{figure}


	\begin{table}[H]
	 \centering
	 \caption{Distribution of number of reviewed movies per publication on rotten tomatoes} 
	 \begin{tabular}{ l | c | c | c | c }
	 \hline
	 &  Min & Max & Mean & Std Dev  \\
	 \hline
	 Top Publications & 0 & 4135 & 97.22 & 454.15 \\
	 Other Publications & 0 & 3224 & 297.78 & 520.28 \\
	 \hline
	 \end{tabular}
	 \end{table}

	 \begin{figure}[H]
	    \centering
	    \includegraphics[width=0.48\textwidth]{plots/plot_r_pub_top.png}
	    \includegraphics[width=0.48\textwidth]{plots/plot_r_pub_oth.png}
	    \caption{INSERT TITLE}
	    \label{fig:r_pub}
	\end{figure}

\subsection{Metacritic}


	\begin{table}[H]
	 \centering
	 \caption{Distribution of number of reviews by users and critics for movies on metacritic}

	 \begin{tabular}{ l | c | c | c | c }
	 \hline 
	 &  Min & Max & Mean & Std Dev  \\
	 \hline
	 Critics with reviews & 0 & 49 & 25.72 & 10.83 \\
	 Users with reviews & 0 & 842 & 21.56 & 50.05 \\
	 \hline
	 \end{tabular}
	 \end{table}

	 \begin{figure}[H]
	    \centering
	    \includegraphics[width=0.48\textwidth]{plots/plot_m_mov_top.png}
	    \includegraphics[width=0.48\textwidth]{plots/plot_m_mov_usr.png}
	    \caption{INSERT TITLE}
	    \label{fig:m_mov}
	\end{figure}


	\begin{table}[H]
	\centering
	 \caption{Distribution of number of reviewed movies by users and critics on metacritic}

	 \begin{tabular}{ l | c | c | c | c }
	 \hline
	 &  Min & Max & Mean & Std Dev  \\
	 \hline
	 Reviews per critic & 56 & 3445 & 1203.36 & 912.50 \\
	 Reviews per user & 1 & 536 & 3.45 & 14.55 \\
	 \hline
	 \end{tabular}
	 \end{table}


	 \begin{figure}[H]
	    \centering
	    \includegraphics[width=0.48\textwidth]{plots/plot_m_crit_top.png}
	    \includegraphics[width=0.48\textwidth]{plots/plot_m_crit_usr.png}
	    \caption{INSERT TITLE}
	    \label{fig:m_crit}
	\end{figure}

\section{Matrix factorization}

	\begin{figure}[H]
	\centering
	\includegraphics[width=0.48\textwidth]{plots/test-i100d1l0.png}
	\includegraphics[width=0.48\textwidth]{plots/test-i100d1l1.png}
	\includegraphics[width=0.48\textwidth]{plots/test-i100d1l3.png}
	\includegraphics[width=0.48\textwidth]{plots/test-i100d1l10.png}
	\caption{Mean squared training and test error over 100 iterations in the stochastic matrix factorization model. Stocastic gradient descent was done using a step size of 0.02. The learned critic matrix was count(critics) by 1, and the learned movie matrix was 1 by count(movies).}
	\label{fig:1}
	\end{figure}


	\begin{figure}[H]
	\centering
	\includegraphics[width=0.48\textwidth]{plots/test-i100d10l0.png}
	\includegraphics[width=0.48\textwidth]{plots/test-i100d10l1.png}
	\includegraphics[width=0.48\textwidth]{plots/test-i100d10l3.png}
	\includegraphics[width=0.48\textwidth]{plots/test-i100d10l10.png}
	\caption{Mean squared training and test error over 100 iterations in the stochastic matrix factorization model. Stocastic gradient descent was done using a step size of 0.02. The learned critic matrix was count(critics) by 10, and the learned movie matrix was 10 by count(movies).}
	\label{fig:10}
	\end{figure}


	\begin{figure}[H]
	\centering
	\includegraphics[width=0.48\textwidth]{plots/test-i100d25l0.png}
	\includegraphics[width=0.48\textwidth]{plots/test-i100d25l1.png}
	\includegraphics[width=0.48\textwidth]{plots/test-i100d25l3.png}
	\includegraphics[width=0.48\textwidth]{plots/test-i100d25l10.png}
	\caption{Mean squared training and test error over 100 iterations in the stochastic matrix factorization model. Stocastic gradient descent was done using a step size of 0.02. The learned critic matrix was count(critics) by 25, and the learned movie matrix was 25 by count(movies).}
	\label{fig:25}
	\end{figure}


	\begin{figure}[H]
	\centering
	\includegraphics[width=0.48\textwidth]{plots/test-i100d40l0.png}
	\includegraphics[width=0.48\textwidth]{plots/test-i100d40l1.png}
	\includegraphics[width=0.48\textwidth]{plots/test-i100d40l3.png}
	\includegraphics[width=0.48\textwidth]{plots/test-i100d40l10.png}
	\caption{Mean squared training and test error over 100 iterations in the stochastic matrix factorization model. Stocastic gradient descent was done using a step size of 0.02. The learned critic matrix was count(critics) by 40, and the learned movie matrix was 40 by count(movies).}
	\label{fig:40}
	\end{figure}


\section{Recommender Systems}

What we are trying to do here is to recommend a critic to a user based on past
critic ratings and past user ratings. In other words, we are trying to build
a recommender system.

Formally, a recommender system takes a set of users $U = \{u_1, ..., u_N\}$, a
set of items $I = \{i_1, ..., i_M\}$, and a sparse matrix of ratings
$R$ of size $N \times M$. If $R_{k,l} > 0$, then user $u_k$ has given item $i_l$ a rating of $R_{k,l}$. A zero rating signifies that the
user has not rated the item. The goal of the system is then to predict the rating that a user would give to an item he has not yet rated. The system can
then recommend any number of items with highest predicted rating.

As described in \cite{Survey05}, recommender systems can be divided in three broad categories: content-based systems, collaborative systems and hybrid systems. 

Content-based recommender systems try to identify item features in order to compare items and recommend similar items to those that the user has rated highly in the past.  For example, if a user consistently rates history books highly, the system can recommend other history books.

Collaborative recommender systems do not try to extract features from the items
to be recommended. Instead, such systems will recommend items that users with similar tastes have rated highly in the past.

Hybrid recommender systems try to combine both approaches in order to improve recommendations.

\section{Further work}


\bibliography{bibliography}{}
\bibliographystyle{plain}

\end{document}